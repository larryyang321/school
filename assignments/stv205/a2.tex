\documentclass[12pt]{article}
\usepackage{amsmath,amssymb,bookmark,graphicx,parskip,textcomp,custom}
\usepackage[margin=.8in]{geometry}
\allowdisplaybreaks
\hypersetup{colorlinks,
    citecolor=black,
    filecolor=black,
    linkcolor=black,
    urlcolor=black
}

\bibliographystyle{plain}

\begin{document}

\title{STV 205 --- Assignment 2}
\author{Kevin James (20463098)}
\date{\vspace{-2ex}Winter 2017}
\maketitle\HRule

\section*{}
Coleman describes Anonymous as a confusing web of inter-connected ideals and beliefs, an often politically-motivated body formed of such a diverse set of goals that it is impossible to pin them down. In this way, they can be seen as a front for a large number of political activists for various issues hidden behind Anonymous' image. "Misinformation about Anonymous abounds" \cite{coleman} and this misinformation provides a smokescreen for any Anon's activism. It is almost irrelevant to discuss Anonymous as a whole but rather provides more value to speak of Anonymous as a group of similarly-structured political activism (or "hactivism") with a large and diverse set of goals.

By organizing in this way, Anonymous can provide a safe haven for political activists and shift the media focus of their group onto the actions undertaken by the members rather than the internal politics of that group. In doing so, Anonymous creates a strong staging ground for the sort of political unrest that helps keep the democratic process strong -- the ability for the democratic constituents to make their voices and unrest known to those in power.

Many of the actions undertaken by Anonymous members are often considered illegal or illegitimate forms of civil disobedience, but it is my belief that this consideration is only due to the "newness" of their techniques, the ability of Anons to use new forms of civil disobediance never seen before. The actions taken by hacktivists often have real-world analogues which have long been considered valid forms of protest; the only difference here is that these techniques have been updated for the digital age.

DDoS tactics are a strong example of this: all across the world, DDoS tactics have been considered illegal \cite{ukddos}, but are these techniques really any different than more physical protesting techniques? It is considered a legal and legitimate form of protest to physically picket in front of business to prevent customers from entering the place of business\dots why, then, is it treated differently to prevent those customers from accessing the business's online presence instead? This dichotomy abounds among the political tools used by hacktivists -- often, the actions undertaken on the internet are considered to be of a completely different category when, in reality, they are analoguous to accepted real-world tools.

Anonymous is trying to address serious problems with democracy, as they see it: lack of privacy, class division, and dishonesty in politics, to name a few of their more hot-button topics. For many of these issues, Anonymous members view the nature of these problems as mostly a symptom of lack of knowledge -- by getting the knowledge of these issues out to the general public, Anons believe they can help turn the tide on these issues. Their strategies, then, which mostly involve attacking highly visible targets and ensuring they have "copious media coverage" \cite{coleman} of those attacks clearly follow these goals and fit the nature of how they perceive the problems they face.

The alternative view, that the issues Anonymous faces can be solved directly rather than through increasing awareness, would also be covered by the actions Anonymous takes; their current attacks, in addition to increasing awareness, also have real effects on their targets and in fact cause organizations to be leery of performing actions they think could attract Anonymous attention -- in effect, making them leery of taking actions which may displease their constituents, as any politician or person in power should be. Anonymous strategies, then, have dual-pronged effect: in addition to increasing awareness of the issues each of their Ops highlights, those Ops themselves help to ensure targets are aware of the unrest around their actions.

Overall, Anonymous itself and hacktivism more generally supports democracy by encouraging the participation and engagement of a larger portion of citizens. Hacktivism is the latest embodiment of the political activism necessary to have a functioning democracy; to consider hacktivism as undermining democracy is to undermine democracy itself. The actions Anonymous undertakes to get closer to this goal are not the illegal lashing outs of a misbehaving sub-culture but are rather the voices of those people whose voices are often ignored speaking up and making themselves known, helping to shape our democracy to more accurately reflect its members.

Though Anonymous has been less active in recent years, their effects can be seen even to the present day. Indeed, the culture of believing those involved in this culture of hacktivism, whistleblowing, and internet politics are a problem rather than the fundamental core of democracy has continued until today \cite{whistleblower}, regularly recieving media attention of the wrong sort.

Anonymous made many issues visible to the average person and helped to increase awareness of those issues, but has not finished in their role; we are still plagued with a socity that functions at its best only when those who are discontent remain quiet. Until this has been changed, we still need those who stand up for their beliefs, in no matter the fashion, just as Anonymous has done.

Word Count: 892 words.

\newpage
\bibliography{a2}

\end{document}
