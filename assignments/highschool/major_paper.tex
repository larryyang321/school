\documentclass[12pt]{article}
\usepackage[margin=1in]{geometry}
\usepackage{parskip}

\bibliographystyle{apalike}

\begin{document}

\title{Corruption in ``The Picture of Dorian Gray''}
\author{Kevin James}
\maketitle

\section*{}
The true measure of a man is not shown through how he presents himself to the outside world or how he appears, and thus can not be determined by an outside source. Only a portrait of such a man's soul could provide an accurate representation, as the soul reflects the thoughts, desires, and external events which have shaped the man. The Picture of Dorian Gray shows a man obsessed with the cultivation of his desires, to the exclusion of all else, and to such a point where his soul's purity suffers and is cruelly disfigured by both his own actions and those of forces seeking to corrupt him.

As Lord Henry states: ``there is no such thing as a good influence\dots all influence is immoral''~\cite[23]{wilde}. Although this is quite the cynical view, it does have merit in that all influence has an effect on the person it is influencing. As this influence is external, and thus not `generated' by the initial purity of an uncorrupted soul, the soul in question would be distorted somewhat by the event. Whether this distortion is beneficial or not, immoral or not, is unrelated to the question of whether or not these influences do occur from all outside sources, but does provide a question useful to the direct analysis of Dorian Gray. Are there both good and bad influences? Surely, their morality is much more difficult to determine than their presence.

In fact, this view is far too narrow to be completely true. A murder who has been presented with external influences in such a way as to cause him to decide to stop his life of crime has undoubtedly been influenced positively\dots in the eyes of the general public. To himself, however, these influences have caused him to change who he is as a person, to become someone he is not. To other murders, secret benefactors, anyone who `gets something' out of the murders, et cetera, this influence may also be construed as negative. The point, then, is that even an influence which seems to have the most positive effect may seem completely different depending on perspective.

The same is true with Dorian Gray. Dorian is influenced negatively, on the whole, as the portrait of his soul shows a ``gradual but definite degradation in his character''~\cite[516]{wainwright}, but the perspective from which this influence seems negative may not be the best perspective to accept. A person such as Basil Hallward may be of the strong opinion that this influence is quite negative, however someone such as Lord Henry would likely have the opposite opinion. To Dorian Gray himself, even, the classification of external influences as being either positive or negative is unlikely to be `cut and dry'.

In my opinion, many of the influences commonly seen as negative by critics of Wilde's work are, in fact, positive influences. Dorian's desire to live forever and to explore all aspects of the human condition in such a way as to provide himself pleasure is not at all a negative desire. Although brought on by Lord Henry, who said himself that ``all influence is immoral''~\cite[23]{wilde}, this desire is, in fact, one which I believe we should all strive for. The basic yearning for an eternal life of experience is not one which is inherently wrong, although it is often construed that way by critics.

Where critics say, however, that ``[this yearning] causes Dorian to be reduced to immoral debauchery'', I have to agree. This distinction, however, is one of importance, in that Dorian's initially idealistic world view and that which comes form him following this view are two completely different things. It could be argued that the external influences which caused Dorian's world view are thus positive, and the only negative aspects of Dorian's personality come from within himself, the only place free from influence.

The fact that this is arguable, however, bring up an important point: if we can't determine absolutely whether an influence is `good' or `bad' from the perspectives of any of the characters in the novel, or from the outside perspective of a literary critic, how can we go about defining certain influences as positive or negative?

We can attempt to use morality, and define for ourselves a set of rules which we can then use to determine the effects of an influence, but even this is subject to perspective-based issues. As all definitions are somewhat arbitrarily stated rules provided through the skewed perspective of an individual, using morality as a basis for the analysis of influence is likely to have the same problems as using a set perspective.

Perhaps the only way, then, to analyze the effects of influence is to look at the very sources seeking to corrupt Dorian, and attempt to determine whether they are seeking to corrupt him or simply happen to corrupt him. Every even a person is presented with in his or her life, after all, influences them, but surely the vast majority of these are accidental. Idle conversation with a stranger may cause one to decide to purchase a more stylish watch, for example, but surely this stranger has not actively set out to cause this decision.

It is important, then, to realize that only those forces that actively seek to corrupt others can be construed as purposefully malicious, and even then only if they, themselves, realize their actions are malicious. As we can not know the mind of these forces, however, we must simply analyze those influences which have had the greatest effect and ignore the purpose of these influences.

These influences are undoubtedly many, but the character of Lord Henry and the ``poisonous [french] book''~\cite[128]{wilde} are without a doubt the most important of them. Both Lord Henry and the french novel (which is almost certainly À Rebours, written by J.-K. Huysmans in 1884) exhibit a similar influence on Dorian: they suggest to him that hedonism is the optimal school of thought, and that ``the only things worth pursuing in life are beauty and fulfillment (sic) of the senses''~\cite{wk}.

Lord Henry's influences are many, and almost certainly have the purpose of corrupting—or, at least, influencing—Dorian. Lord Henry actively seeks out Dorian for conversation, considering him a ``young man of extraordinary personal beauty''~\cite[3]{wilde}, ``a wonderful creation''~\cite[23]{wilde}, and thus worthy of hearing his thoughts. These thoughts are the initial reasons why Dorian decides to sell his soul in exchange for eternal life—especially when Lord Henry tells Dorian that ``beauty is a form of genius-- \dots higher, indeed, than genius''~\cite[24]{wilde}. He also subtly pushes Dorian toward the acceptance of his desires, telling him that ``the only way to get rid of a temptation is to yield to it''~\cite[20]{wilde}.

Upon hearing Lord Henry's description of hedonism and his thoughts about beauty, Dorian realizes that even his own beauty will eventually fade, and thus expresses his desire to see the portrait age instead of himself. As such, he becomes eternally beautiful and causes the sins of his soul to be expressed upon a portrait, to be examined physically.

This external manifestation of his soul's corruption is vital to the story, as it is virtually impossible to analyze the state of one's soul without any basis for such an analysis. With the portrait, however, we can see the physical results of the disfigurement of his soul, and thus can examine the effects of various actions with a great deal more specificity. It is for this reason, too, that Dorian locks the portrait in a dusty room—for he would never want to share ``the secret of [his] own soul''~\cite[7]{wilde}. In fact, it can be said the ``his portrait is rather the more real thing os the two [of them]''~\cite[114]{hawthorne}, for the effects of influence shown on Dorian's soul are characterized with a great deal more precision that Dorian himself.

Upon realizing that his wish for eternal youth has come true, Dorian immerses himself within the teachings of Lord Henry and plunges himself into debauched acts which provide him with pleasure at the cost of his soul's purity, becoming ``a wonderful study''~\cite[76]{wilde} ``promis[ing] rich and fruitful results''~\cite[503]{wainwright} for Lord Henry to watch. In this way, Lord Henry embodies the traditional role of the devil, for he ``lives to witness the destruction of every other person in [the novel]''~\cite[113]{hawthorne}. Dorian also reads the novel presented to him by Lord Henry, À Rebours, and his lifestyle begins to reflect that of the main character in that novel.

This novel teaches many of the same things as Lord Henry: it idolizes hedonism and the search for happiness, considering such things to be life's sole true purpose. It throws out the traditional ideology surrounding morality, and instead tells the reader to live life with no regard to societal customs, to achieve spiritual and emotional happiness through any means possible, and to have ``a wholesome dislike of the common-place''~\cite[115]{pater}. In effect, to act as if ``there are no laws''~\cite[125]{ellman} of any form governing one's behaviour. In this way, the book certainly has the purpose of corrupting Dorian—although not aiming solely for the corruption of Dorian Gray, it is obvious that the book intends to influence all of it's readers.

In the case of Dorian Gray, we can see that the largest influences on him [Dorian] are negative, and corrupt him through their presence. Dorian falls into the same mode of thinking as Lord Henry, and idolizes the ideas presented in À Rebours, even while he condemns it for being ``a poisonous book''~\cite[128]{wilde}. He changes his lifestyle to reflect the achievement of his desires, instead of the maintenance of his soul.

The achievement of desires is a strong theme within this novel, for it is contrasted by the idea of purity in all ways and presented to Dorian as a crucial choice he must make. Dorian—due to the influences provided to him by Lord Henry and À Rebours—chooses to follow his desires, to the detriment of his soul.

Hedonism and decadent literature are the perfect embodiment of `desire', for they emphasize this over the importance of maintaining a pure soul. In contrast, spiritual fulfilment and tradition seem to be the antithesis, for they emphasize purity of the soul above all else, even at (or, perhaps, especially at) the loss of the achievement of one's desires. These two ideas are in constant opposition, working to gain dominance over one's thoughts and actions, and to contort the soul in whatever way they see fit.

The search for desires, which Dorian accepts as his raison d'être, is portrayed as contorting the soul in such a way that it becomes disfigured due to the sins one is often required to commit whilst achieving one's desires.

The idea of `disfigurement' seems to be defined much the same way as Lord Henry describes influence—that is, no distortion can ever be a positive change. By this definition, it of course follows that all influence is negative, et cetera, however, I believe it is important to note that there can be positive disfigurement and negative disfigurement. Much in the same way as the morality of an influence is subjective, so, too, is the result of a distortion. If there can be a good influence, the change this influence causes upon one's soul must also be seen as a good thing, at least within the same perspective.

The disfigurement of one's soul is also commonly related to a person's beauty. Some people feel that beauty is internal, and that the true measure of one's beauty is their soul's disfigurement (or lack thereof) Such people would likely think that ``vice and crime make people coarse and ugly''~\cite[116]{pater}. Others, however, feel that beauty is a measure of external beauty, unrelated to the `look' of one's soul. This distinction is crucial in Dorian Gray, for Dorian decides that external beauty is far more important than internal beauty, and thus cares not whether anything distorts his internal beauty. He even allows his murder of Basil Hallward to ``dehumanize him by robbing him of his ordinary human passion''~\cite[86]{riquelme}, in a way giving his soul's disfigurement an external projection.

It follows, then, that Dorian Gray cares much more for the achievement of his desires, for their affect on his soul is not one which he is concerned about, despite the physical vision of his soul which he has access to. It makes one wonder, however. Although often portrayed in a negative light, is it true that hedonism and such ideals are truly negative? In a more `normal' case, where one does not have access to a vision of their own soul, does it really matter how much one's soul has been disfigured?
\newpage

\bibliography{major_paper}
\end{document}
